% Vorlage für Abschlussarbeiten
% Hier werden allgemeine Einstellungen festgelegt

\documentclass[%
	a4paper,%							A4 Papier
	oneside,%	einseitig (linker und rechter Seitenrand sind gleich groß)
	bibliography=totoc,%	Literaturverzeichnis nummeriert mit ins Inhaltsverzeichnis einfügen
	%listof=totoc,%				Abbildungs- und Tabellenverzeichnis ins Inhaltsverzeichnis einfügen
	numbers=noenddot,%		hinter der Gliederungsnummer soll kein Punkt gesetzt werden (siehe Duden)
	parskip=half,%				europäischer Satz mit Abstand zwischen Absätzen
]{scrartcl}

% ---- Metadaten ----
% Titel der Arbeit
\title{Title der Arbeit}	
% Autor
\author{Vorname Name}		
% Datum
\date{\today}	         
% Datei Name
\newcommand{\name}{Dokument.pdf}
% Name Dozent
\newcommand{\dozent}{Name}
% Abgabedatum
\newcommand{\abgabedatum}{15.12.2021}
% Kopieren den Titel und den Autor zur Verwendung in der Kopf- und Fußzeile.
\makeatletter
	\let\runauthor\@author
	\let\runtitle\@title
\makeatother

% ---- Packages ----

% --- Font ---
% Eingabecodierung, deutsche Umlaute oder die akzentuierten Zeichen sind verfügbar 
% und können direkt eingegeben werden
% siehe auch http://de.wikipedia.org/wiki/UTF8
\usepackage[utf8]{inputenc}			
% Ausgabeschriftart von LaTeX festlegen
% siehe auch http://de.wikibooks.org/wiki/LaTeX-Schnellkurs:_Erste_Schritte
\usepackage[T1]{fontenc}					
% Schriftart
\usepackage[scaled]{helvet}
% Text Symbole
\usepackage{textcomp}	

% --- Language ---
% Paket für Deutsche Sprache (Übersetzungen von Chapter zu Kapitel, 
% richtige Umlaute, richtige % Silbentrennung)
% siehe auch http://de.wikipedia.org/wiki/Babel-System
\usepackage[english,ngerman]{babel}
\usepackage[ngerman, num]{isodate}
% Entferne Absatzeinzug 
\usepackage{parskip}

% --- Last Page ---
% Erweitert die Befehle mit \pageref{LastPage}.
\usepackage{lastpage} 						

% --- Layout ---
\usepackage{geometry} 							
\usepackage{fancyhdr} 						
\usepackage{graphicx}	

% --- Mindmap ---
\usepackage{tikz}								
\usetikzlibrary{mindmap}

\usepackage{lscape}

% --- Debug ---
% Segmente anzeigen
%\usepackage{showframe}					

% ---- Style ----
\geometry{
    % Legt einen Rand oben fest.
	tmargin=25mm,							
	% Legt einen Rand unten fest.
	bmargin=25mm,							
    % Legt einen Rand links vom Inhalt fest.
	lmargin=30mm,		
	% Legt einen Rand rechts vom Inhalt fest.
	rmargin=20mm,						
	headheight=34.78244pt
}
% Römisch Arabisch Aufzählung wechseln
\newcommand{\sectionnumbering}[1]{%
\setcounter{section}{0}%
\renewcommand{\thesection}{\csname #1\endcsname{section}}}
% Blendet die oberste Linie aus.
\renewcommand{\headrulewidth}{0pt}
% Blendet die Untere Linie aus.
\renewcommand{\footrulewidth}{0pt}	
% Setzt die Standardschriftfamilie auf eine sans serif
\renewcommand\familydefault{\sfdefault}	
% Path zu den Bildern
\graphicspath{{./img/}} 

% Kopf und Fusszeilen Format
\pagestyle{fancy}							
\fancyhf{}					
% Logo im Header					
\rhead{\includegraphics[width=2cm]{logo}}		 
\lhead{\fontsize{8}{10} \selectfont \runtitle}	
\cfoot{\fontsize{8}{10} \selectfont \runauthor}	
\lfoot{\fontsize{8}{10} \selectfont \today}	
\rfoot{\fontsize{8}{10} \selectfont Seite \thepage\ von \pageref{LastPage}}
% Ende der Datei